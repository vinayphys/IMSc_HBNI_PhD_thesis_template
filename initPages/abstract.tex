\thispagestyle{plain}
\doublespacing
\vskip 1.0cm
%
\centerline{{\bf{\large Abstract}}}
%
This thesis explores different aspects of thermal and mechanical response of glassy materials using large-scale computer simulations. In particular, we study the response of model glass-forming systems under thermal perturbation in the form of a temperature gradient or under mechanical perturbation in the form of shear or a combination of both.  We start with a study of supercooled as well as glassy states to a thermal gradient and observe increasing concentration inhomogeneity with lowering of temperature, via the interplay of mass and heat transport.  We also devise a heating-cooling protocol, where an appropriate thermal gradient pulse is applied, to show that it is possible to tune the concentration and density profiles of the glassy states obtained at the end. Such protocol can be used to produce glassy materials having inhomogeneous structure. Further, we study the mechanical response of these inhomogeneous glassy states to understand the pathway to failure, where we observe that the timescale for emergence of the non-equilibrium steady state depends upon the thermal processing, which consequently affects the formation of shear-bands in the transient regime. Next, we study the Poiseuille flow of a confined soft glass under two different thermalization protocols, resulting in the presence or absence of thermal gradient across the channel, which affects the local flow properties of the confined glass differently, indicating the possibility of tuning the rheology. At the end, for a  model binary mixture with large size bidispersity among the particles, known to show large separation in structural relaxation timescale of large and small species, we demonstrate finite-size effects in the measurements of single particle dynamical quantities due to finite interdiffussion even when the large particles are in dynamical arrest. Moreover, via detailed microscopic and macroscopic analysis, we demonstrate that the shear response of this binary mixture shows an interesting  rigidity scenario, viz. a finite yield stress even when the small particles are still mobile. Our analysis also shows that the increasing presence of smaller particles has a softening effect on the mixture.