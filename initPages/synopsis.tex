\thispagestyle{plain}
\doublespacing
\chapter*{Synopsis}

\vskip 1.0cm   
{\bf {\Large Introduction}}
\vskip 0.5cm
Glassy materials in various forms: metallic alloys, window glass, biological tissues, colloidal suspensions etc, are integral part of our life. The art of manufacturing glasses is known to human civilization for centuries. With the progress of human race, use of glasses in various applications also developed. It is difficult to imagine life without glasses because they are important component for various applications that we use on daily basis, like lens in the camera, screen of the smartphone, glass as thermostat in bottle etc. These materials have structure very similar to high temperature liquids but the structural relaxation exceeds experimentally available timescales. There have been continuous scientific investigations \cite{angell1995formation, berthier2011theoretical, binder2011glassy} to understand the hidden information in the structure of glassy materials which lead to ultra slow dynamics where viscosity of the material become $10^{14}$ times the viscosity of the normal liquid, this means effective behaviour as solid. In general, the understanding of emergence of rigidity in such materials is important to design new materials with applications. Glassy systems have many unique properties attributed to them: heterogeneous dynamics, change in properties with age (aging), mechanical memory, etc. Although, many of these properties have been harnessed for various applications but deeper understanding of these would led to development of new applications.

The goal of this thesis is to study the thermal and mechanical response of glassy systems \cite{nicolas2018deformation, schuh2007mechanical}. We apply thermal perturbation in the form of temperature gradient and mechanical perturbation in the form of shear. Also, in some instances a combination of thermal and mechanical perturbations is applied. Study of such thermal or mechanical response will not only lead to development of various important applications but will also help in understanding the out-of-equilibrium behaviour of amorphous system, in general. Thermal processing of glasses in various forms has already been used to tune structure and hence properties of glasses \cite{schuh2007mechanical,hufnagel2015cryogenic,ketov2015rejuvenation}. Also, study of mechanical response of glassy materials has helped in understanding the flow of complex liquids \cite{nicolas2018deformation,bonn2017yield} and developing memory storage devices \cite{keim2019memory}.

Due to significant increase in computer capabilities in last few decades, numerical simulation has emerged as an important tool to address a variety of problems related to glassy physics. We also take the help of large scale computer simulations to put a model system  under the influence of a thermal gradient or mechanical drive then spatio-temporal response is recorded with microscopic details, soon after applying the perturbation and also after long time in steady state if it can be reached. The strength of the perturbation is always one control parameter but other parameters, depending on the model that we choose, may also be varied to enable the study of glassy material in supercooled state or in glassy state. Nonequilibrium molecular dynamics simulation has been used throughout this thesis as a major technique  to explore various questions.  We have used homegrown codes and also open source softwares \cite{plimpton1995fast, stukowski2009visualization} to setup the simulations. As a part of different studies under thesis, many analytical approach and their codes are developed and  tested, which we also expect to be useful for various other research projects. Now I would like to outline a brief summary of main results that we got while answering various questions relevant to the theme of this thesis.

\vskip 1.0cm
{\bf {\large Thermal response of glassy system}}
\vskip 0.3cm
Our objective is to probe the behaviour of a glassy material when an external temperature gradient is applied to it. The detailed understanding of such behaviour is important to describe and develop various natural and manmade systems: magmatic differentiation during pertrogenesis, leakage of trapped nuclear waste in glass, compositional inhomogeneity in the formation of metallic alloys, etc. We have performed extensive molecular dynamics simulations using a model binary glass-forming mixture \cite{kob1995testing} to understand the response to a thermal gradient \cite{vaibhav2020response}. When a thermal gradient is imposed, two types of current viz. thermal and mass currents are set up in the system. In steady state, mass current goes to zero and we get steady non-uniform concentration profiles along the direction of thermal gradient. In this study, we undertake a microscopic analysis of the steady state, to understand the response of the two species within the binary mixture.  The steady state in the linear response regime can be characterized by measuring a quantity called Soret coefficient which is the ratio of the thermal diffusion coefficient and interdiffusion coefficient. We show that the Soret coefficient in the glassy mixture increases as the glass transition is approached. Further, the response becomes nonlinear if the strength of the gradient is increased beyond a limit and this limit is dependent upon the thermal state of the supercooled liquid. So, we construct a state diagram, across a wide range of temperatures, to characterise the linear-nonlinear response of the supercooled glassy liquid. We have also measured the thermal response of the system in the glassy state, where we observe the absence of any significant change in concentration profiles within the observed time window, as expected from the arrested dynamics of the system below glass transition temperature. Finally, by applying a heating-cooling protocol where an appropriate thermal gradient pulse is switched on for a finite duration and then switched off (very similar to laser heating or laser ablation), we have shown that it is possible to tune the concentration profiles of the glassy states obtained at the end. Thermal gradient pulse causes local melting, leading to density inhomogeneity and after removal of the gradient the inhomogeneous density profile freezes. Such protocol can be used to produce glassy materials having inhomogeneous structure. We discuss more about this in next section.

\vskip 1.0cm
{\bf {\large Mechanical response of thermally processed glass}}
\vskip 0.3cm

As we have already discussed above, imposing a thermal gradient to a liquid mixture can initiate transport processes resulting in a spatially non-uniform density and concentration profiles \cite{vaibhav2020response}. Now we want to understand the mechanical response of glassy states which have been exposed to a thermal gradient pulse leading to a density inhomogeneity. Such study can be helpful to understand the pathway to failure in systems like metallic glasses where inhomogeneity is inherently present due to manufacturing process. It would be useful to explore a protocol, if it exists, which could generate a tuned density profile to have a control over the shear bands formed. Using large scale computer simulations, we expose glassy states below mode-coupling temperature to thermal gradient which induces a spatial density inhomogeneity \cite{vaibhavPrep}. The order of inhomogeneity developed in the sample depends on the size of the gradient that we apply and also the exposure time for which the gradient is on. Then, the shear-response of these thermally processed samples is studied by deforming the sample at fixed shear-rates. We observe that the timescale for emergence of the non-equilibrium steady state under shear depends upon the thermal processing, which consequently affects the formation of shear-bands in the transient regime. Further, we identify the case where a glassy state is exposed to a particular gradient for sufficient interval to allow us to establish a connection between density inhomogeneity and shear-band formation. Overall, this study should motivate to fine-tune engineered processes that help in improving desired properties of materials, for example ductility of metallic glasses, by minimizing the chances of failure.

\vskip 1.0cm
{\bf {\large Poiseuille flow of soft glass: role of thermalization protocol}}
\vskip 0.2cm
A flow through a channel or pipe due a pressure gradient is called Poiseuille flow and this is common to many natural systems and applications like microfluidic devices, 3D printing etc, involving soft glasses. Glasses are known to exhibit finite yield stress and understanding the flow properties of these materials, which makes a large part of the amorphous family, in such setup can help us to develop various applications. In this particular work \cite{vaibhav2021influence}, we perform a numerical study to develop a microscopic understanding of how two different temperature control mechanisms with varying forcing strength affect the Poiseuille flow of a soft glass. The first one is a \textit{wall thermostat} where we use the confining walls to thermalize the system. In this case, atoms building the confining walls are vibrating and maintained at a fixed temperature. This eventually leads to a steady non-uniform temperature profile across the channel because of the continuous heat transfer from the center of the channel to the walls. In the second method, a thermostat (\textit{DPD thermostat}) is directly applied to the confining fluid while walls are frozen, resulting no temperature variation across the channel. We compare the steady flow properties of the glassy material in the two thermalization protocols and also with Couette flow under similar conditions. In the case of DPD thermostatted rheology, flow curves at different forcing strength overlap with bulk flow curve obtained via Couette flow at given temperature. However, since there is a presence of temperature gradient in wall thermostat case, flow-curves show increasing deviation from bulk behaviour as the forcing strength is varied. We have also compared the transient behaviour of the flow, where we find that the flow starts early in wall thermostat case compared with the DPD thermostat, again due to the presence of the thermal gradient. We also do a comparative study of these flow properties by changing the changing channel width. To conclude, this study will help in developing flow based applications where Poiseuille setup of yield-stress materials can be controlled via appropriate thermal control.

\vskip 1.0cm
{\bf {\large Glassy binary mixture with large size bidispersity: interdiffusion and rheology}}
\vskip 0.5cm

In nature or in our daily life, we see a variety of materials where the constituent particles can have a large dispersity in sizes. Most of the studies that have happened to explore various aspects of glassy systems, use a model binary mixture where the size of the particles are comparable. Only a few studies have been performed in recent years using binary mixtures with large size asymmetry. In particular, the rheological properties of such mixtures have not been much explored. 

For our study, we have considered a model binary mixture with large size bidispersity (diameter ratio between larger and smaller particles is $\approx 2.85$), where particles interact via purely repulsive potential. This mixture is known to show increasing separation in relaxation timescales of the larger and smaller species, as the density of the system is increased \cite{voigtmann2009double}. It has been shown that the dynamics of bigger particles has been fitted using mode-coupling theory which gives a critical density where these particles show glass transition. At the same time, the smaller particles dynamics show finite diffusion and a possible glass transition only at very high density.

 In the first part of this work, our main focus is to understand the collective transport process, viz. interdiffusion which is a measure of relaxation of concentration fluctuation in the mixture \cite{vaibhav2022finite}. We measure single particle diffusion and the collective interdiffusion, which leads to the surprising observation that around the mode-coupling critical density, the selfdiffusion coefficient of the larger particles starts to follow a rather weak dependence on density. We demonstrate that this behaviour is due to the finite interdiffusion within the system, even though the larger particles undergo glass transition. This happens because smaller particles have finite diffusion even at high densities, as a result center of mass of bigger species also diffuse due to momentum conservation. We elucidate that this results in finite-size effects in the self-diffusion of the bigger species. Further, we propose that if the calculation of single particle quantities of larger and smaller species are done in their respective center of mass frames, then the correct dynamical behaviour of larger species can be extracted.

In the second part of this work \cite{vaibhav2022rheological}, we have studied the shear response of the same binary mixture with large size ratio. Since this mixture has wide timescale separation among the bigger and smaller species, it is very interesting to analyse the competition between these intrinsic timescales and the external timescale introduced via shear. We  impose shear  at different rates on the system and study the response by performing extensive molecular dynamics simulations. The macroscopic response is observed by measuring stress and viscosity as function of shear rate, at different densities. For small densities, the behaviour is like Newtonian liquid over a large window of shear-rates, which contracts and ultimately appears to vanish as the density of system is increased. Near the mode-coupling critical density of the larger species, the system shows onset of rigidity evidenced by appearance of dynamic yield stress in small shear-rate regime. Further, we measure single particle quantities like mean-squared displacement and van-Hove correlations to understand the behaviour of the system at microscopic level. From these micro and macro measurements, we conclude a density dependent response of the two species in the system: (i) at very small density, neither larger nor smaller particles respond to the shear; (ii) at intermediate densities, only larger species respond to the shear; and (iii) at very high density, much above mode-coupling critical density of larger species, both species start responding to the shear. To investigate the role of smaller species in the rheology, we perform simulations varying the composition of the system and conclude that adding smaller particles makes system softer, i.e. reduces the measured viscosity.

\vskip 1.0cm
{\bf {\large Summary}}
\vskip 0.3cm
In summary, via this thesis, we have developed microscopic understanding of how glassy system, in general, responds to various thermal and mechanical disturbances, using the technique of computer simulations. At first, a thermal gradient is applied to a model glass-former, which responds with the formation of spatial compositional inhomogeneity. Then these inhomogeneous states are sheared to explore an engineered protocol to establish a connection between the pathway to failure and density inhomogeneity. In an another part of the thesis, we study soft glasses in Poiseuille flow setup, where we explore the flow properties in the presence or absence of a thermal gradient. In the last part of the thesis, we study the self and collective transport processes of a binary mixture with large size asymmetry and use these inputs to study rheology of the same mixture to develop microscopic understanding that how rigidity sets in.  Overall, the work reported in this thesis helps us in advancing our understanding of the thermomechanical properties of different glassy systems.

