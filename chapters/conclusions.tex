\pagestyle{fancy}
\fancyhf{}
\renewcommand{\headrulewidth}{0pt}
\fancyfoot[C]{\leftmark}
\fancyhead[R]{\thepage}
\doublespacing

\chapter{Conclusions and outlook}\label{chap7}
In this chapter, we first provide some conclusive remarks, summarising the overall understanding of the theme of this thesis. The second part of the chapter highlights some interesting directions in which the different projects of the thesis can be extended in future.


\section{Conclusions}
In this thesis, we have studied different aspects of the thermal and mechanical response of glassy systems, using large-scale computer simulations. Via numerical investigations, we have studied the response of model glass-forming systems, to understand the generic behaviour of glassy materials, under thermal perturbation in the form of a temperature gradient or under mechanical perturbation in the form of shear or a combination of both.


In Chapter-\ref{chap1}, we introduced glassy materials and their properties, along with a brief and general overview of the thermal and mechanical response of these materials, as needed for our studies. This chapter is followed by chapter-\ref{chap2}, where we discussed the numerical techniques used in other chapters to simulate the glassy systems and their responses. In particular, we have discussed the technique of molecular dynamics simulations which is used to study equilibrium and nonequilibrium phenomena. Also, we discussed how the recently developed technique of swap Monte Carlo in combination with molecular dynamics can be used to equilibrate glassy systems beyond the conventional limits.


From Chapter-\ref{chap3} to Chapter-\ref{chap6}, we have discussed four different projects where responses to thermal and mechanical perturbations are reported. Temperature gradient (as thermal perturbation) is present in different forms in the studies reported in Chapter-\ref{chap3}, Chapter-\ref{chap4} and Chapter-\ref{chap5}, while mechanical perturbation is the focus of Chapter-\ref{chap4}, Chapter-\ref{chap5} and Chapter-\ref{chap6}. Hence, in Chapter-\ref{chap4} and Chapter-\ref{chap5}, studies involving both thermal and mechanical perturbation are present. In Chapter-\ref{chap4}, at first, a thermal perturbation is applied and then removed and then we study the response to an imposed mechanical perturbation. Chapter-\ref{chap5} discusses and analyses a situation where thermal and mechanical perturbations are present together. In this manner, we have studied the different thermo-mechanical responses of glassy systems.

In Chapter-\ref{chap3}, we studied the response of a model glass-forming liquid to a thermal gradient. The coupling of heat and mass transport is studied in the supercooled limit via the Soret coefficient, as the system is cooled towards the glass transition. It is concluded that the increased coupling of the two transport processes at low temperatures is due to the slow interdiffusion process. Also, a two-dimensional response diagram is constructed, which shows the linear response regime contracts with the decrease in temperature. Further, we studied the thermal response in the glassy state where no large-scale mass transport, characteristic of the Soret effect, is observed. Finally, using samples in the glassy state, we studied their response to a thermal gradient pulse where we observed that the thermodynamic pulses can be used to produce glassy samples with spatial heterogeneity.

The idea of generating inhomogeneous glassy samples using thermal gradient pulses, tested in Chapter-\ref{chap3}, is further developed and extended in Chapter-\ref{chap4}. It is shown how the spatial heterogeneity of glassy samples can be controlled by tuning the strength and exposure time of the thermal gradient pulse. Further, these glassy samples of varying degrees of heterogeneity were subjected to a shear deformation to understand the macro and microscopic mechanism of failure. It was observed that the steady state of homogeneous flow in heterogeneous glassy samples is delayed. At the microscopic scale, we plotted mobility maps to analyse the shear band formation; we observed that the microscopic mechanism of failure is controlled by the heterogeneity which in turn is controlled by the temperature gradient.

In Chapter-\ref{chap5}, we studied the Poiseuille flow of a model glass-forming liquid under two different temperature control protocols. The first protocol used a thermal wall where atoms that form the wall vibrate to maintain  the target temperature, while the other protocol used a thermostat directly applied to the confined fluid keeping the wall atoms frozen. We studied the Poiseuille flow by varying the forcing strength (which causes the flow in the channel) under the two thermalization protocols, one where there is a temperature gradient (wall thermostat) and the other where there is no gradient (direct thermostat). In particular, the flow curves showed different branches for different forcing strengths in the case of wall thermostats while in the case of direct thermostatting, all flow curves overlapped with each other when forcing strengths were varied. Thus, the local rheology is determined by the nature of the thermalizing process. Also, the transient behaviour was compared in the two thermalization protocols, where we observed that the flow starts early when thermostatting happens via walls, which could be utilised in applications.

In the last study reported in Chapter-\ref{chap6}, a very special type of binary mixture was used where the size ratio between the bigger and smaller species is larger than the conventional glassy mixtures and sufficient enough to create a large separation in their relaxation time scales. We studied the structural and dynamical properties of the mixture to understand the interdiffusion behaviour, and observed a finite-size effect in the dynamical properties caused by the asymmetry in the dynamics of the two species. We provided the resolution to the finite-size effects by suggesting the calculation of dynamical quantities for different types of species in their respective center of mass coordinates. Keeping this resolution in mind, the rheological properties of the mixture were studied to understand the interplay of timescale introduced via shear to the intrinsic relaxation timescales of the mixture. We measured the macro-response which shows a density-dependent response, which can be understood via the microscopic dynamics of the two species. Also, the microscopic analysis of the emergence of rigidity in the mixture confirmed the existence of mobile small particles when the material becomes rigid. Finally, a separate analysis was done to understand the contributions of the two species to the total stress that develops when the mixture is sheared, which established the fact that small particles have a small but important contribution in determining the mechanical properties of the material.


\section{Outlook}
Different studies performed in this thesis under the title of "{\em Thermo-mechanical response of glassy systems}" can be extended in many directions, leading to some interesting results. Here we discuss some possible avenues of exploration.

In this thesis, we have reported the response of a glass-forming system to a thermal gradient. But the model glass-former that we have used is non-bonded and non-network forming. Therefore, a possible extension of the project to glass-forming systems that form networks, e.g. silica, which have useful industrial applications due to their special transport properties. Similarly, the idea developed in Chapter-\ref{chap3} to design inhomogeneous glassy materials using a temperature gradient pulse, can be further developed to design a material with a specific type of inhomogeneities such that the pathway of mechanical failure could be controlled, especially in the context of micro-machining.  Again, in the same scope of aiming of developing applications involving yield stress materials, the Poiseuille flow study that we have performed in Chapter-\ref{chap5} where the temperature of the flowing liquid can be controlled via a vibrating wall, can be extended to study a channel flow with specific thermal environments to obtain the desired flow behaviour. These investigations can be extended to other flow protocols as well.

In Chapter-\ref{chap6}, we have studied interdiffusion and rheological properties of a binary mixture where the size ratio between bigger and smaller species is relatively larger than the conventional mixtures. These studies have been done for one specific size ratio and one temperature. Extending these investigations to a wider range of asymmetries and thermal scales is the next step, besides understanding the microscopic behaviour related to the onset of flow at a local scale. 

Overall, there are several directions to explore involving thermo-mechanical properties of glass, based on the studies reported in this thesis.
